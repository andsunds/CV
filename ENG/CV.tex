%% start of file `template.tex'.
%% Copyright 2006-2015 Xavier Danaux (xdanaux@gmail.com).
%
% This work may be distributed and/or modified under the
% conditions of the LaTeX Project Public License version 1.3c,
% available at http://www.latex-project.org/lppl/.


\documentclass[12pt,a4paper, sans, swedish, english]{moderncv}        % possible options include font size ('10pt', '11pt' and '12pt'), paper size ('a4paper', 'letterpaper', 'a5paper', 'legalpaper', 'executivepaper' and 'landscape') and font family ('sans' and 'roman')
\pdfoutput=1

\usepackage[T1]{fontenc}
\usepackage[utf8]{inputenc}
\usepackage{babel}
\usepackage{helvet}
\usepackage{footmisc}
\usepackage{lastpage}
\rfoot{\textit{\small{\thepage/\pageref{LastPage}}}}

% moderncv themes
\moderncvstyle{classic}

% adjust the page margins
\usepackage[scale=0.80, left=1.cm,right=2.cm]{geometry}
%\setlength{\hintscolumnwidth}{3cm}                % if you want to change the width of the column with the dates
%\setlength{\makecvtitlenamewidth}{10cm}           % for the 'classic' style, if you want to force the width allocated to your name and avoid line breaks. be careful though, the length is normally calculated to avoid any overlap with your personal info; use this at your own typographical risks...

\setlength{\hintscolumnwidth}{2.5cm}

% personal data
\name{\Huge Andréas}{\Huge Sundström}
\title{{\normalsize Swedish personal number: 940310-3410}}                               % optional, remove / comment the line if not wanted
\address{Blåsutgatan 9B}{SE-414 56, Gothenburg}{Sweden}% optional, remove / comment the line if not wanted; the "postcode city" and "country" arguments can be omitted or provided empty
\phone[mobile]{+46\,76-849\,34\,63}                   % optional, remove / comment the line if not wanted; the optional "type" of the phone can be "mobile" (default), "fixed" or "fax"
\email{sundstrom.andreas@gmail.com}                               % optional, remove / comment the line if not wanted

% bibliography adjustements (only useful if you make citations in your resume, or print a list of publications using BibTeX)
%   to show numerical labels in the bibliography (default is to show no labels)
%\makeatletter\renewcommand*{\bibliographyitemlabel}{\@biblabel{\arabic{enumiv}}}\makeatother
%   to redefine the bibliography heading string ("Publications")
%\renewcommand{\refname}{Articles}

% bibliography with mutiple entries
%\usepackage{multibib}
%\newcites{book,misc}{{Books},{Others}}
%----------------------------------------------------------------------------------
%            content
%----------------------------------------------------------------------------------
\begin{document}

%-----       resume       ---------------------------------------------------------
\makecvtitle

\section{Education}
% arguments 3 to 6 can be left empty
\cventry{2018--(2022)}{PhD in Physics}{Chalmers University of Technology}{Gothenburg}{}{%
Title of thesis: \textit{Collisional effects and attosecond
  diagnostics in laser-generated plasmas}.}
%
\cventry{2013--2018}{Diploma engineer in Engineering Physics}{Chalmers University of Technology}{Gothenburg}{}{}
%
\cventry{2013--2018}{MSc in Physics and astronomy}{Chalmers University of Technology}{Gothenburg}{}{%
Title of thesis: \textit{Effects of electron trapping and ion collisions on electrostatic shocks}}
%
\cventry{2016--2017}{Exchange studies}{University of Waterloo}{Waterloo, ON, Canada}{}{%
Forth year of studies in Canada.}
%
\cventry{2013--2016}{BSc in Engineering Physics}{Chalmers University of Technology}{Gothenburg}{}{%
Title of thesis: \textit{Brownsk rörelse i celler hos partiklar och
  filament} (Eng: \emph{Brownian motion inside cells for particles and
filaments})}


%\section{Master thesis}
%\cvitem{title}{\emph{Title}}
%\cvitem{supervisors}{Supervisors}
%\cvitem{description}{Short thesis abstract}

\section{Work experience}
% \subsection{Vocational}
\cventry{2018--(2022)}{PhD student}{Chalmers University of
  Technology}{Gothenburg}{}{%
  The research projects that I have been pursuing during my PhD have
  all been related to numerical and analytical modeling of the
  interaction between the plasma and electromagnetic fields of
  high-intensity laser pulses. As a part of my PhD, I have been
  working closely with collaborators in Lund, Princeton and Paris; the
  latter as a part of a one-month research visit to Sorbonne
  University in March 2022.\\
  During my PhD, I have written several research articles, and
  presented our work (both as posters and oral
  presentations) at international conferences.\\
  I have also had the opportunity to teach mechanics and vector-field
  theory to the Engineering physics students.  }
% 
\cventry{Aug 2017}{Student Research Assistant}{Chalmers University of
  Technology}{Gothenburg}{}{At the Department of Electrical
  Engineering, Biomedical electromagnetics research group. I and an
  other student developed a theoretical base for focusing microwaves
  into tumors.}
% 
\cventry{2015--2016}{Teaching Assistant}{Chalmers University of
  Technology}{Gothenburg}{}{At Mathematical Sciences, tutoring
  mathematics and Matlab for first year undergraduate students.}
% 
\cventry{Jun--Aug 2014}{Lab assistant}{SKF}{Gothenburg}{}{At the
  Manufacturing Development Centre (MDC), preparing and testing
  samples of heat treated steel.}
% 

\newpage
\section{Non-profit work}
\cventry{2014--2020}{Assisting the Swedish national physics competition}{}{}{}{
  I have been assisting the organization of the Swedish national high
  school physics competition ``Wallenberg Physics Prize''. My tasks have
  been to help coming up with ideas for the experimental part of the
  national finals, and teaching the Swedish IPhO (International
  Physics Olympiad) team at a weekend intensive course in experimental
  skills, of which I took on full responsibility since 2016.}
%
\cventry{2019}{Team co-leader at NBPhO for the Swedish team}{}{}{}{
   I was one of two team leaders for the Swedish team at the
   Nordic--Balitic Physics Olympiad 2019, held in Tallin, Estonia. As
   the team leaders, we looked after the team of approximately 20
   high-school students during the competition. We also had
   responsibility of coming up with and debating exam problems, as
   well as correcting the exams.}
%
\cventry{2016 \& 2017}{Observer at IPhO for the Swedish team}{}{}{}{
  I have followed the Swedish IPhO team to the Olympiads of 2016 and
  2017 as an observer, which entails translating and marking the exam
  questions for the Swedish participating high school students. }

%\newpage
\section{Awards and Scholarships}
\cventry{2021}{Scholarship from the Adlerbertska research fund}{}{}{}{
  %Amount: 49\,000~SEK,
  For a one-month research visit to Sorbonne University. The
  scholarship is awarded based on a research proposal submitted to
  Kungliga Vetenskaps- och Vitterhets-Samhället i Göteborg (KVVS). }
%
\cventry{2019}{John Ericsson Medal}{}{}{}{
The John Ericsson Medal is awarded yearly to the five graduates
with the highest average grades across all of Chalmers. }
%
\cventry{2016}{Scholarship from Kamrathjälpsfonden}{}{}{}{
  %Amount: 17\,500~SEK,
  For studies in Canada. Managed by the Swedish
  Association of Graduate Engineers. }
% 
\cventry{2016}{Scholarship from the Adlerbertska trust funds
  (Swe. Albrektska stiftelerna)}{}{}{}{%Amount: 10\,000~SEK,
  Based on rate of study and grades. Managed by Chalmers
  Scholarships.} %
%
\cventry{2015}{IPT, 7th place}{}{}{}{International Physicists'
  Tournament, a team of six students from each university solves 17
  physics problems and present their solutions at the tournament.} %
%
\cventry{2015}{Best experimental work}{}{}{}{
  In the \textit{Experimental physics} course for engineering physics
  students at Chalmers University of Technology. A work which later
  resulted in the article \textsl{Measuring g using a rotating liquid
    mirror: enhancing laboratory learning}, see Publications.}
%
\cventry{2013}{Wallenberg Physics Prize, 2nd place (individual)}{}{}{}{The Swedish
  national physics competition for high school students.}
%
\cventry{2013}{Wallenberg Physics Prize, 1st place in the Team
  competition}{}{}{}{The Swedish national physics competition for high
  school students. The team competition is comprised of the three best
  results from each participating school.}
%
\cventry{2013}{IPhO, Honorable mention}{}{}{}{
  International Physics Olympiad, over 80~countries participating with
  over 400~students in total -- each country sends up to five
  students.}
%
\cventry{2013}{Scholarship from the Folke and Märta Wiesel scholarship
  fund}{}{}{}{%Amount: 8\,000~SEK,
  For ``\textsl{the graduating student} [...] \textsl{with the best
    record in physics}'' from Hvitfeldtska high school, Gothenburg.}
%
\cventry{2012}{Wallenbergs Physics Prize, 7th place (individual)}{}{}{}{The Swedish
  national physics competition for high school students.}
%
\cventry{2012}{IPhO, participated}{}{}{}{
  International Physics Olympiad, over 80~countries participating with
  over 400~students in total -- each country sends up to five
  students.}
\begin{flushright}\footnotesize
\textit{Any claims to an award can be backed by an official certificate.}
\end{flushright}


%\newpage
\section{Languages}
\cvitemwithcomment{Swedish}{Fluent}{Mother tongue}
\cvitemwithcomment{English}{Fluent}{Both written and spoken daily in current work}


\section{Computer skills}
% \cvdoubleitem{Programming}{C, Python, Matlab, Bash,}{Software}{Matlab,
%   \LaTeX, Emacs,}
% \cvdoubleitem{languages}{Java}{}{UNIX terminal, Particle-in-Cell
%   codes}

\cventry{Programming languages}{Python, C, C++, \LaTeX{}, \textsc{Matlab}}{}{}{Java, Bash}{%
  As a part of my PhD I have extensively used a massively parallel
  ``particle-in-cell'' (PIC) plasma-simulation code on large
  computational clusters. The PIC code is written in C++ with a user
  interface in Python; I have developed several post-processing tools
  implemented in \textbf{Python} for the output from the PIC
  simulations. I have also implemented a special diffusion module in
  the \textsc{dream}\footnotemark{} fusion-research code, written in
  \textbf{C++}. \\
  \textsc{Matlab} was extensively used during my BSc and MSc studying
  period, as well as the first year of my PhD. As my MSc diploma
  project I used \textbf{\textsc{Matlab}} to developed and implemented
  a package for calculating the electrostatic potential of a shock
  wave in a plasma. }
%\stepcounter{footnote}
\footnotetext{GitHub:
  \url{https://github.com/chalmersplasmatheory/DREAM}\quad
  Documentation: \url{https://ft.nephy.chalmers.se/dream/}}
%
\cventry{Other software}{Emacs, Git, UNIX terminal}{}{}{}{}

\section{Leisure activities and hobbies}
\cvitem{Sailing}{I sail on the Swedish West coast for a least two week
  every summer.}
%
\cvitem{Electronics}{I have a small electronics lab, in which I
  sometimes build smaller electronic projects, e.g.\ a central clock,
  timed by the 50\,Hz mains frequency, that control an old secondary
  clock via alternating pulses every minute. }
%
\cvitem{Classic car}{I own a 1965 Volvo Amazon, for the express
  purpose of having something ``where I can get my hands dirty''. I
  enjoy working on the car as a break from the computer-bound life in
  the office.}
%
\cvitem{Photography}{I like to shoot with my old Hasselblad
  and Leica (analog) cameras. I also develop black\,\&\,white film at
  home.}


\newpage
\section{Publications}
\newcommand{\doi}[1]{\textsc{doi}:~\href{http/doi.org/#1}{#1}} 
%
\cvitem{2022}{\textbf{A. Sundström}, M. Grech, I. Pusztai~\& C. Riconda.
2022 \textsl{Stimulated-Raman-scattering amplification of attosecond
XUV pulses and application to local in-depth plasma-density
measurement}. %
Submitted to Physical Review E
\href{http://arxiv.org/abs/2207.06761}{http://arxiv.org/abs/2207.06761}} 
%
\cvitem{2022}{\textbf{A. Sundström}, I. Pusztai, P. Eng-Johnsson~\&
  T. Fülöp. 2022 \textsl{Attosecond dispersion as a diagnostics tool
    for solid- density laser-generated plasmas}. %
  Journal of Plasma Physics \textbf{88} (2), 905880\,211,
  \doi{10.1017/S0022377822000307}}

%
\cvitem{2020}{\textbf{A. Sundström}, E. Siminos, L. Gremillet~\&
  I. Pusztai, 2020 \textsl{Collisional effects on the ion dynamics in
    thin-foil targets driven by an ultraintense short pulse laser}. %
  Plasma Physics and Controlled Fusion \textbf{62} (8), 085\,015,
  \doi{10.1088/1361-6587/ab9a62}}
%
\cvitem{2020}{\textbf{A. Sundström}, E. Siminos, L. Gremillet~\&
  I. Pusztai, 2020 \textsl{Fast collisional electron heating and
    relaxation with circularly polarized ultraintense short-pulse
    laser}. %
  Journal of Plasma Physics \textbf{86}, 755860\,201,
  \doi{10.1017/S0022377820000264}} 
% 
\cvitem{2019}{\textbf{A. Sundström}, J. Juno, J.M.\ TenBarge \&
  I. Pusztai. 2019 \textsl{Effect of a weak ion collisionality on the
    dynamics of kinetic electrostatic shocks}. %
  Journal of Plasma Physics \textbf{85}, 905850\,108 
  \doi{10.1017/S0022377819000023}}
%
\cvitem{2016}{\textbf{A. Sundström} and T. Adawi. 2016 \textsl{Measuring g
    using a rotating liquid mirror: enhancing laboratory learning}. %
  Physics Education \textbf{51}, 053004 
  \doi{10.1088/0031-9120/51/5/053004}}

\end{document}


%  LocalWords:  Matlab Wallenbergs fysikpris IPhO

%%% Local Variables:
%%% mode: latex
%%% TeX-master: t
%%% End:
